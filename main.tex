\documentclass{article}
\usepackage{graphicx}
\usepackage[center]{caption}
% If you're new to LaTeX, here's some short tutorials:
% https://www.overleaf.com/learn/latex/Learn_LaTeX_in_30_minutes
% https://en.wikibooks.org/wiki/LaTeX/Basics

% Formatting
\usepackage[utf8]{inputenc}
\usepackage[margin=1in]{geometry}
\usepackage[titletoc,title]{appendix}

% Math
% https://www.overleaf.com/learn/latex/Mathematical_expressions
% https://en.wikibooks.org/wiki/LaTeX/Mathematics
\usepackage{amsmath,amsfonts,amssymb,mathtools}
\usepackage{bm}
\usepackage{siunitx}

% Images
% https://www.overleaf.com/learn/latex/Inserting_Images
% https://en.wikibooks.org/wiki/LaTeX/Floats,_Figures_and_Captions
\usepackage{graphicx,float}

% Tables
% https://www.overleaf.com/learn/latex/Tables
% https://en.wikibooks.org/wiki/LaTeX/Tables

% Algorithms
% https://www.overleaf.com/learn/latex/algorithms
% https://en.wikibooks.org/wiki/LaTeX/Algorithms
\usepackage[ruled,vlined]{algorithm2e}
\usepackage{algorithmic}

% Code syntax highlighting
% https://www.overleaf.com/learn/latex/Code_Highlighting_with_minted
\usepackage{minted}
\usemintedstyle{borland}

% References
% https://www.overleaf.com/learn/latex/Bibliography_management_in_LaTeX
% https://en.wikibooks.org/wiki/LaTeX/Bibliography_Management
\usepackage{biblatex}
\addbibresource{references.bib}

\DeclareMathOperator{\taninv}{tan\,inverse}

% Title content
\title{ENPM 673 Project 4}
\author{Arjun Srinivasan Ambalam, Praveen Menaka Sekar, Arun Kumar Dhandayuthabani}
\begin{document}

\maketitle

% Introduction and Overview
\section{Inverse Compositional Algorithm}
There is a huge computational cost in re-evaluating the Hessian in every iteration of the Lucas-Kanade algorithm proposed originally.To reduce this cost,there is an updated process named Inverse Compositional Algorithm.In this method,various  parameters are updated after a couple of iterations. The Hessian is considered to be relatively constant and pre-computed.The goal of the inverse compositional algorithm is the same as the Lucas-Kanade algorithm; i.e. to minimize:

 the cost of the inverse compositional algorithm is

 iteration rather than

for the Lucas-Kanade algorithm, a substantial saving.

The inverse compositional algorithm iteratively applies Equations (12) and (13)


\section{Steps for Inverse Compositional KLT}

\subsection{Pre-Computation}
\begin{enumerate}
    \item Evaluate the gradient 5T of the template T(x)
    \item Evaluate the Jacobian ∂W/∂p at (x; 0).
    \item Compute the Steepest descent images 5T(∂W/∂p)
    \item Compute the Hessian matrix
\end{enumerate}
\subsection{Iterative Steps}
\begin{enumerate}
    \item  Warp I with W(x; p) to compute I(W(x; p))
    \item  Compute the error in the image I(W(x; p)) − T(x)
    \item  Compute ∆p
    \item Update the warp W (x; p) ← W (x; p) ◦ W (x; ∆p)power(-1) until k∆pk ≤ e
\end{enumerate}


\end{document}